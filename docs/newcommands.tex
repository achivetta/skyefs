%-- place any standard commands/environments here to get included in
%-- documents.  When you include this file, you should do it before
%-- the \begin{document} tag.

%%%%%%%%%%%%%%%%%%%%%%%%%%%%%%%%%%%%%%%%%%%%%%%%%%%%%%%%%%%%%%%%
% Special font and spacing formatting

%-- Terms...  Use this to introduce a term in the paper.
\newcommand{\term}[1]{\emph{#1}}

%-- Provides fixed width font for code snippets.
% \newcommand{\code}[1]{\texttt{\textbf{#1}}}
% \newcommand{\code}[1]{\textsc{\texttt{#1}}}
\newcommand{\code}[1]{\texttt{#1}}

%-- Commands: e.g., WRITE command.
\newcommand{\command}[1]{{\textsc \MakeLowercase{#1}}}

%-- Jiri caption
\newcommand{\minicaption}[2]{\caption[#1]{\textbf{#1.} #2}}

% Unit(s)
%-- unit: 4KB -> 4\unit{KB}
\newcommand{\unit}[1]{\,#1} % thin space followed by units
%-- Units on numbers: 4KB -> \units{4}{KB}
\newcommand{\units}[2]{#1\unit{#2}}
% Specific units...
\newcommand{\micros}{\ensuremath{\mu}{s}}

%-- Inline title
\newcommand{\inlinesection}[1]{\smallskip\noindent{\textbf{#1.}}}

%%%%%%%%%%%%%%%%%%%%%%%%%%%%%%%%%%%%%%%%%%%%%%%%%%%%%%%%%%%%%%%%
% Editting commands

%-- For notes about things that need to be fixed.
\newcommand{\fix}[1]{~{\LARGE\ensuremath{\star}}~\textbf{#1}~{\LARGE\ensuremath{\star}}~}

\newcommand{\bcut}{\marginpar{\ensuremath{\bigvee}}}
\newcommand{\ecut}{\marginpar{\ensuremath{\bigwedge}}}

%-- Draft watermark in upperleft corner
\newcommand{\reviewtimetoday}[2]{\special{!userdict begin
/bop-hook{gsave 20 710 translate 45 rotate 0.8 setgray
/Times-Roman findfont 12 scalefont setfont 0 0   moveto (#1) show
0 -12 moveto (#2) show grestore}def end}}


%%%%%%%%%%%%%%%%%%%%%%%%%%%%%%%%%%%%%%%%%%%%%%%%%%%%%%%%%%%%%%%%
% Table and Figure formatting

% Table format
% inter-row spacing for tables
\renewcommand{\arraystretch}{1.0}
% Control spacing of ``double lines'' (default is 2pt)
% Can set this within braces but before \begin{table}...
%\setlength{\doublerulesep}{0.2mm} % into a single think line
% \setlength{\doublerulesep}{3pt} % obvious spacing.
\setlength{\tabcolsep}{1pt} % default is 6pt

% Table commands
% font size for tables
%\newcommand{\tablefontsize}{\scriptsize}
%\newcommand{\tablefontsize}{\footnotesize}
\newcommand{\tablefontsize}{\small}
%\newcommand{\tablefontsize}{}

% \usepackage{tabularx}
% Using L, C, R as the column type will left, center, or right justify text.
\newcolumntype{L}{X}
\newcolumntype{C}{>{\centering\arraybackslash}X}
\newcolumntype{R}{>{\raggedleft\arraybackslash}X}

% Figure commands
\newcommand{\figurewidth}{\columnwidth}

% algorithm / pseudo-code
\newcommand{\pseudosize}{\scriptsize}
\newcommand{\pseudowidth}{\columnwidth}
\newcommand{\pseudoskip}{\smallskip}
%\newcommand{\pseudoskip}{}

% pseudo-code commands
\newcommand{\ccomment}[1]{ /$*$ #1 $*$/}	
\newcommand{\rccomment}[1]{\hfill\ccomment{#1}} % right aligns the comment

%%%%%%%%%%%%%%%%%%%%%%%%%%%%%%%%%%%%%%%%%%%%%%%%%%%%%%%%%%%%%%%%
% Math. Uses package ``amsthm''.

%-- Influenced by the ACM journal tex template
%\newtheorem{theorem}{Theorem}[section]
%\newtheorem{conjecture}[theorem]{Conjecture}
%\newtheorem{corollary}[theorem]{Corollary}
%\newtheorem{proposition}[theorem]{Proposition}
%\newtheorem{lemma}[theorem]{Lemma}
%\theoremstyle{definition} % This currently does nothing...
%\newtheorem{definition}[theorem]{Definition}
%\newtheorem{observation}[theorem]{Observation}
%\newtheorem{remark}[theorem]{Remark}

%%-- Referencing Lemma, Observation, Definition
%\newcommand{\thmref}[1]{Theorem~\ref{#1}}
%\newcommand{\lemref}[1]{Lemma~\ref{#1}}
%\newcommand{\defref}[1]{Definition~\ref{#1}}

%%-- Examine different math fonts 
%%-- \usepackage{amsmath}
%%-- \usepackage[mathscr]{eucal} % \mathcal and \mathscr for fancy calligraphy.
%\newcommand{\mathfonts}[1]{ \text{math:~} {#1},\\ \text{mathsf:~} \mathsf{#1},\\ \text{mathcal:~} \mathcal{#1}\\ \text{mathscr:~} \mathscr{#1},\\ \text{mathrm:~} \mathrm{#1},\\ \text{mathbb:~} \mathbb{#1},\\ \text{mathbf:~} \mathbf{#1},\\ \text{mathtt:~} \mathtt{#1}}


%%%%%%%%%%%%%%%%%%%%%%%%%%%%%%%%%%%%%%%%%%%%%%%%%%%%%%%%%%%%%%%%
% Lists
\newenvironment{outline}
{
	\renewcommand{\baselinestretch}{1.0}
	\footnotesize
    \begin{list}
    {--} % Leave empty for nothing, or replace bullet with any other symbol.
    {
        \setlength{\partopsep}{0in}
        \setlength{\topsep}{0in}
        \setlength{\parsep}{0in}
        \setlength{\itemsep}{0in}
        \setlength{\leftmargin}{0.3in}
    }
}
{
    \end{list}
}


% Lists with single spacing
\newenvironment{my_enumerate}{
\begin{enumerate}
  \setlength{\partopsep}{0in}
  \setlength{\topsep}{0in}
  \setlength{\itemsep}{1pt}
  \setlength{\parskip}{0pt}
  \setlength{\parsep}{0pt}
}{\end{enumerate}}

\newenvironment{my_description}{
\begin{description}
  \setlength{\partopsep}{0in}
  \setlength{\topsep}{0in}
  \setlength{\itemsep}{1pt}
  \setlength{\parskip}{0pt}
  \setlength{\parsep}{0pt}
}{\end{description}}

\newenvironment{my_itemize}{
\begin{itemize}
  \setlength{\partopsep}{0in}
  \setlength{\topsep}{0in}
  \setlength{\itemsep}{1pt}
  \setlength{\parskip}{0pt}
  \setlength{\parsep}{0pt}
}{\end{itemize}}

\newenvironment{my_list}{
\begin{list}{}{
%  \setlength{\itemindent}{\leftmargin}
  \setlength{\topsep}{0in}
  \setlength{\partopsep}{0in}
  \setlength{\topsep}{0in}
  \setlength{\itemsep}{1pt}
  \setlength{\parskip}{0pt}
  \setlength{\parsep}{0pt}
}}{\end{list}}



\hyphenation{OpenAFS}
\hyphenation{FSVA}
